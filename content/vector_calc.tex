\section{Vector Calculus}

\subsection*{Scalar Fields}
$f : \mathbb{R}^n \to \mathbb{R}$

\subsection*{Vector Fields}
$\ul{v} : \mathbb{R}^n \to \mathbb{R}^m$

\subsection*{Differentiation of Vector Fields}
Let $\ul{a}, \ul{b}, \ul{c}$ be vector fields and $\varphi$ be a scalar field, then
\begin{align*}
	\frac{d}{dt} (\varphi \ul{a})= & \frac{d\varphi}{dt}  \ul{a}+ \varphi \frac{d\ul{a}}{dt}\\
	\frac{d}{dt} (\ul{a} \cdot \ul{b})= &  \frac{d\ul{a}}{dt} \cdot \ul{b} +  \ul{a} \cdot \frac{d\ul{b}}{dt}\\
	\frac{d}{dt} (\ul{a} \times \ul{b})= &  \frac{d\ul{a}}{dt} \times \ul{b} +  \ul{a} \times \frac{d\ul{b}}{dt} \\
	\frac{d}{dt} ( \ul{a}(\varphi(t) ) )= & \frac{d\ul{a}}{d\varphi} \frac{d\varphi}{dt} \\
\end{align*}


\subsection*{Differential Operators}
For a scalar field $f : \mathbb{R}^n \to \mathbb{R}, \ul{x} \mapsto f(\ul{x})$ the gradient is
\[
	\text{grad } f(\ul{x})  = \colvec{ \frac{\del f}{\del x_1} \\ \vdots \\ \frac{\del f}{\del x_n} } 
\]
and gives the direction of \ul{steepest ascent}


For a vector field $\ul{v} : \mathbb{R}^n \to \mathbb{R}^n$ the divergence is defined as 
\[
	\text{div } \ul{v}(\ul{x}) = \sum_{j=1}^n \frac{\del v_j}{\del x_j}(\ul{x})
\]
and gives the \ul{source density} at a point

For a vector field $\ul{v} : \mathbb{R}^3 \to \mathbb{R}^3$ the curl is defined as 
\[
	\text{curl }(\ul{v}(x,y,z)) = \colvec{ \frac{\del v_3}{\del y} - \frac{\del v_2}{\del z} \\
						 \frac{\del v_1}{\del z} - \frac{\del v_3}{\del x}  \\
						 \frac{\del v_2}{\del x}  - \frac{\del v_1}{\del y} }
\]
and gives the \ul{vortex strength}


The Laplace operator for a scalar field $f : \mathbb{R}^n \to \mathbb{R}, \ul{x} \mapsto f(\ul{x})$ gives
\[
	\Delta f(\ul{x})  = \nabla \cdot \nabla f(\ul{x}) = \sum_{j=1}^n \frac{\del^2 f}{\del^2 x_j}(\ul{x})
\]


Expressed with the $\nabla$ operator we get the representations

 \begin{align*}
	\text{grad } f = & \nabla f \\
	\text{div } \ul{v} = & \nabla \cdot \ul{v}\\
	\text{curl } \ul{v} = &  \nabla \times \ul{v} \\
	\Delta =& \nabla \cdot \nabla \\
\end{align*}


\subsection*{Computation Rules for Differential Operators}
The three operators grad, div and curl are linear. 
Furthermore
\begin{align*}
	\text{grad }(f_1 f_2) = & f_1 \text{grad }( f_2) + f_2 \text{grad }(f_1 )\\
	\text{grad }F(f) = & F'(f) \text{grad }(f) \\
	\text{curl } \ul{v}_1 + \ul{v}_2 = &  \text{curl } \ul{v}_1 + \text{curl } \ul{v}_2\\
	\text{div } (\ul{v}_1 \times \ul{v}_2) = & \ul{v}_2 \cdot \text{curl }\ul{v}_1 - \ul{v}_1 \cdot \text{curl }\ul{v}_2 \\
\end{align*}

Rules for the interactions between vector- and scalarfields:
\begin{align*}
	\text{div } f\ul{v} = &  \ul{v} \text{grad }( f) + f \text{div } \ul{v}\\
	\text{curl } f \ul{v} = &  f \text{curl }\ul{v}  -  \ul{v} \times \text{grad }(f)\\
\end{align*}

Rules for the concatenation of the operators:
\begin{align*}
	\text{div }\text{curl } \ul{v} = &  0 \\
	\text{curl }\text{grad } f = &  \ul{0} \\
	\text{div }\text{grad } f = & \Delta f \\
	\text{curl }\text{curl } \ul{v} = & \text{grad }  \text{div } \ul{v} - \Delta \ul{v} \\
	\text{div } (\ul{v}_1 \times \ul{v}_2) = & \ul{v}_2 \cdot \text{curl }\ul{v}_1 - \ul{v}_1 \cdot \text{curl }\ul{v}_2 \\
\end{align*}






















\begin{comment}


\begin{align*}
	\text{grad }(f_1 f_2) = & f_1 \text{grad }( f_2) + f_2 \text{grad }(f_1 )\\
	\text{grad }F(f) = & F'(f) \text{grad }(f) \\
	\text{div } f\ul{v} = &  \ul{v} \text{grad }( f) + f \text{div } \ul{v}\\
	\text{div } \ul{v} = & \nabla \cdot \ul{v}\\
	\text{curl } \ul{v}_1 + \ul{v}_2 = &  \text{curl } \ul{v}_1 + \text{curl } \ul{v}_2\\
	\text{curl } c\ul{v} = &  c \text{curl } \ul{v} \\
	\text{curl } f \ul{v} = &  f \text{curl }\ul{v}  -  \ul{v} \times \text{grad }(f)\\
	\text{div }\text{curl } \ul{v} = &  0 \\
	\text{curl }\text{grad } f = &  \ul{0} \\
	\text{div }\text{grad } f = & \Delta f \\
	\text{curl }\text{curl } \ul{v} = & \text{grad }  \text{div } \ul{v} - \Delta \ul{v} \\
	\text{div } (\ul{v}_1 \times \ul{v}_2) = & \ul{v}_2 \cdot \text{curl }\ul{v}_1 - \ul{v}_1 \cdot \text{curl }\ul{v}_2 \\
\end{align*}

\end{comment}

1. Wenn $F \wedge G$ eine Tautologie ist, dann ist $F$ eine Tautologie und $G$ auch.
2. Umgekehrt: Sind $F$ und $G$ Tautologien, dann ist auch $F \wedge G$ eine.
\emph{Beweis.}
1. Annahme: $F \wedge G$ sei eine Tautologie.
Dann: Für jede Belegung $B$ wertet $F \wedge G$ zu wahr aus.
Dann: Das ist nur der Fall, wenn sowohl $F$ als auch $G$ (für jedes $B$) zu wahr auswerten.
Dann: Für jede Belegung $B$ wertet $F$ zu wahr aus. Und:
Für jede Belegung $B$ wertet $G$ zu wahr aus.
Dann: $F$ ist Tautologie und $G$ ist Tautologie.
2. Annahme: $F$ ist Tautologie und $G$ ist Tautologie.
Dann: Für jede Belegung $B_1$ wertet $F$ zu wahr aus. Und: Für jede Belegung $B_2$ wertet $G$ zu wahr aus.
Dann: Für jede Belegung $B$ wertet $F \wedge G$ zu wahr aus.
Dann: $F \wedge G$ ist eine Tautologie.
\subsection*{Äquivalenz und Folgerung}
$p\equiv q$ gilt genau dann, wenn sowohl $p\models q$ als auch $q\models p$ gelten. \emph{Beweis.}
$p\equiv q$ GDW $p\Leftrightarrow q$ ist Tautologie nach Def. von $\equiv$
GDW $(p\Rightarrow q) \wedge (q\Rightarrow p)$ ist Tautologie
GDW $(p\Rightarrow q)$ ist Tautologie und $(q\Rightarrow p)$ ist Tautologie
GDW $(p\models q)$ gilt und $q\models p$ gilt.
\subsection*{Substitution}
Ersetzt man in einer Formel eine beliebige Teilformel $F$ durch eine logisch äquivalente
Teilformel $F'$, so verändert sich der Wahrheitswerteverlauf der Gesamtformel nicht.
Man kann Formeln also vereinfachen, indem man Teilformeln durch äquivalente
(einfachere) Teilformeln ersetzt.
\subsection*{Universum}
Die freien Variablen in einer Aussagenform können durch Objekte aus einer als
Universum bezeichneten Gesamtheit wie $\mathbb{N},\mathbb{R},\mathbb{Z},\mathbb{Q}$ ersetzt werden.
\subsection*{Tautologien}
$(p\wedge q)\Rightarrow p$\text{ bzw. }$p\Rightarrow (p\vee q)$\\
$(q\Rightarrow p)\vee (\neg q\Rightarrow p)$\\
$(p\Rightarrow q)\Leftrightarrow (\neg p\vee q)$\\
$(p\Rightarrow q)\Leftrightarrow (\neg q\Rightarrow\neg p)$ \hfill\text{(Kontraposition)}\\
$(p\wedge (p\Rightarrow q))\Rightarrow q$ \hfill\text{(Modus Ponens)}\\
$((p\Rightarrow q)\wedge (q\Rightarrow r))\Rightarrow (p\Rightarrow r)$\\
$((p\Rightarrow q)\wedge (p\Rightarrow r))\Rightarrow (p\Rightarrow (q\wedge r))$\\
$((p\Rightarrow q)\wedge (q\Rightarrow p))\Leftrightarrow (p\Leftrightarrow q)$
\subsection*{Nützliche Äquivalenzen}
Kommutativität:\\
$(p \wedge q) \equiv (q \wedge p)$\\
$(p \vee q) \equiv (q \vee p)$\\
Assoziativität:\\
$(p \wedge (q \wedge r)) \equiv ((p \wedge q) \wedge r)$\\
$(p \vee (q \vee r)) \equiv ((p \vee q) \vee r)$\\
Distributivität:\\
$(p \wedge (q \vee r)) \equiv ((p \wedge q) \vee (p \wedge r))$\\
$(p \vee (q \wedge r)) \equiv ((p \vee q) \wedge (p \vee r))$\\
Idempotenz:\\
$(p \wedge p) \equiv p$\\
$(p \vee p) \equiv p$\\
Doppelnegation:\\
$\neg (\neg p) \equiv p$\\
de Morgans Regeln:\\
$\neg (p \wedge q) \equiv ((\neg p) \vee (\neg q))$\\
$\neg (p \vee q) \equiv ((\neg p) \wedge (\neg q))$\\
Definition Implikation:\\
$(p \Rightarrow q) \equiv (\neg p \vee q)$\\
Tautologieregeln:\\
$(p \wedge q) \equiv p$\hfill (falls $q$ eine Tautologie ist)\\
$(p \vee q) \equiv q$\\
Kontradiktionsregeln:\\
$(p \wedge q) \equiv q$\hfill (falls $q$ eine Kontradiktion ist)\\
$(p \vee q) \equiv p$\\
Absorptionsregeln:\\
$(p \wedge (p \vee q)) \equiv p$\\
$(p \vee (p \wedge q)) \equiv p$\\
Prinzip vom ausgeschlossenen Dritten:\\
$p \vee (\neg p) \equiv w$\\\
Prinzip vom ausgeschlossenen Widerspruch:\\
$p \wedge (\neg p) \equiv f$
\subsection*{Äquivalenzen von quant. Aussagen}
Negationsregeln:\\
$\neg\forall x:p(x)\equiv\exists x:(\neg p(x))$\\
$\neg\exists x:p(x)\equiv\forall x:(\neg p(x))$\\
Ausklammerregeln:\\
$(\forall x:p(x)\wedge\forall y:q(y))\equiv\forall z:(p(z)\wedge q(z))$\\
$(\exists x:p(x)\wedge\exists y:q(y))\equiv\exists z:(p(z)\wedge q(z))$\\
Vertauschungsregeln\\
$\forall x\forall y:p(x,y)\equiv\forall y\forall x:p(x,y)$\\
$\exists x\exists y:p(x,y)\equiv\forall y\exists x:p(x,y)$
\subsection*{Äquivalenzumformung}
Wir demonstrieren an der Formel $\neg (\neg p \wedge q) \wedge (p \vee q)$, wie man mit Hilfe der
aufgelisteten logischen Äquivalenzen tatsächlich zu Vereinfachungen kommen kann:\\
$\phantom{{}\equiv{}} \neg (\neg p \wedge q) \wedge (p \vee q)$\\
$\equiv (\neg (\neg p) \vee (\neg q)) \wedge (p \vee q)$\hfill de Morgan\\
$\equiv (p \vee (\neg q)) \wedge (p \vee q)$\hfill Doppelnegation\\
$\equiv p \vee ((\neg q) \wedge q)$\hfill Distributivtät v.r.n.l.\\
$\equiv p \vee (q \wedge (\neg q))$\hfill Kommutativtät\\
$\equiv p \vee f$\hfill Prinzip v. ausgeschl. Widerspruch\\
$\equiv p$\hfill Kontradiktionsregel





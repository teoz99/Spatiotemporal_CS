\section{Vector Calculus}

\subsection*{Scalar Fields}
$f : \mathbb{R}^n \to \mathbb{R}$

\subsection*{Vector Fields}
$\ul{v} : \mathbb{R}^n \to \mathbb{R}^m$

\subsection*{Differentiation of Vector Fields}
Let $\ul{a}, \ul{b}, \ul{c}$ be vector fields and $\varphi$ be a scalar field, then
\[
\begin{aligned}
	\frac{d}{dt} (\varphi \ul{a})= & \frac{d\varphi}{dt}  \ul{a}+ \varphi \frac{d\ul{a}}{dt}\\
	\frac{d}{dt} (\ul{a} \cdot \ul{b})= &  \frac{d\ul{a}}{dt} \cdot \ul{b} +  \ul{a} \cdot \frac{d\ul{b}}{dt}\\
	\frac{d}{dt} (\ul{a} \times \ul{b})= &  \frac{d\ul{a}}{dt} \times \ul{b} +  \ul{a} \times \frac{d\ul{b}}{dt} \\
	\frac{d}{dt} ( \ul{a}(\varphi(t) ) )= & \frac{d\ul{a}}{d\varphi} \frac{d\varphi}{dt} \\
\end{aligned}
\]

\subsection*{Differential Operators}
For a scalar field $f : \mathbb{R}^n \to \mathbb{R}, \ul{x} \mapsto f(\ul{x})$ the gradient is
\[
	\grad f(\ul{x})  = \colvec{ \frac{\del f}{\del x_1} \\ \vdots \\ \frac{\del f}{\del x_n} } 
\]
and gives the direction of \ul{steepest ascent}


For a vector field $\ul{v} : \mathbb{R}^n \to \mathbb{R}^n$ the divergence is defined as 
\[
	\div \ul{v}(\ul{x}) = \sum_{j=1}^n \frac{\del v_j}{\del x_j}(\ul{x})
\]
and gives the \ul{source density} at a point

For a vector field $\ul{v} : \mathbb{R}^3 \to \mathbb{R}^3$ the curl is defined as 
\[
	\curl(\ul{v}(x,y,z)) = \colvec{ \frac{\del v_3}{\del y} - \frac{\del v_2}{\del z} \\
						 \frac{\del v_1}{\del z} - \frac{\del v_3}{\del x}  \\
						 \frac{\del v_2}{\del x}  - \frac{\del v_1}{\del y} }
\]
and gives the \ul{vortex strength}


The Laplace operator for a scalar field $f : \mathbb{R}^n \to \mathbb{R}, \ul{x} \mapsto f(\ul{x})$ gives
\[
	\Delta f(\ul{x})  = \nabla \cdot \nabla f(\ul{x}) = \sum_{j=1}^n \frac{\del^2 f}{\del^2 x_j}(\ul{x})
\]


Expressed with the $\nabla$ operator we get the representations
\[
 \begin{aligned}
	\grad f = & \nabla f \\
	\div \ul{v} = & \nabla \cdot \ul{v}\\
	\curl \ul{v} = &  \nabla \times \ul{v} \\
	\Delta = & \nabla \cdot \nabla \\
\end{aligned}
\]

\subsection*{Computation Rules for Differential Operators}
The three operators grad, div and curl are linear. 
Rules for one active operator:
\[
\begin{aligned}
	\grad (f_1 f_2) =& f_1 \, \grad f_2 + f_2 \, \grad f_1\\
	\Leftrightarrow \nabla (f_1 f_2) = & f_1 \, \nabla  f_2 + f_2 \, \nabla f_1\\
	\\
	\grad F(f) =& F'(f) \, \grad f \\
	\Leftrightarrow \nabla F(f) = & F'(f) \, \nabla f \\
	\\
	\div (\ul{v}_1 \times \ul{v}_2) =& \ul{v}_2 \cdot \curl \ul{v}_1 - \ul{v}_1 \cdot \curl \ul{v}_2 \\
	\Leftrightarrow \nabla \cdot (\ul{v}_1 \times \ul{v}_2) = &  \ul{v}_2 \cdot (\nabla \times \ul{v}_1) - \ul{v}_1 \cdot (\nabla \times \ul{v}_2) \\
\end{aligned}
\]

Rules for the interactions between vector- and scalarfields:
\[
\begin{aligned}
	\div f \ul{v} = &  \ul{v} \, \grad f + f \div \ul{v}\\
	\Leftrightarrow \nabla \cdot (f\ul{v}) = &  \ul{v} \, \nabla f + f (\nabla \cdot \ul{v})\\
	\\
	\curl f \ul{v} = &  f \curl \ul{v}  -  \ul{v} \times \grad f\\
	\Leftrightarrow \nabla \times (f \ul{v}) = &  f (\nabla \times\ul{v})  -  \ul{v} \times \nabla f\\
\end{aligned}
\]


Rules for the concatenation of the operators:
\[
\begin{aligned}
	\div \curl \ul{v} = &  0 \\
	\Leftrightarrow \nabla \cdot ( \nabla \times\ul{v} ) = &  0 \\
	\\
	\curl \grad f = &  \ul{0} \\
	\Leftrightarrow \nabla \times \nabla f = & \ul{0}\\
	\\
	\div \grad f = & \Delta f \\
	\Leftrightarrow \nabla \cdot \nabla f = &  \Delta f\\
	\\
	\curl \curl \ul{v} = & \grad  \div \ul{v} - \Delta \ul{v} \\
	\Leftrightarrow \nabla \times (\nabla \times \ul{v}) =  & \nabla  (\nabla \cdot \ul{v}) - \Delta \ul{v}\\
\end{aligned}
\]



\subsection*{Flux ($\Phi$)}

\textbf{Q}: Let $\ul{v}$ be the velocity field of a flow;
how much fluid flows through a surface $S$ per unit time in direction $\ul{n}$?

\textbf{A}: Split $S$ into infinitesimal area elements $dS$. 
Because they are infinitesimal, the $dS$ are flat and $\ul{v}$ is homogeneous within a single $dS$.
The total flow through a single $dS$ is
\[
	d\Phi = \ul{v} \cdot \ul{n} dS
\]

The flux is the “sum” over all infinitesimal $dS$ and thus given by the integral
\[
	\Phi = \int_S \ul{v} \cdot \ul{n} dS
\]





\subsection*{Work ($W$)}

Another fundamental quantity in modeling is the work (a flow of energy) done by
a vector field $\ul{v}$ along a line path $L$ with beginning $A$ and end $B$

\textbf{Q}: Let $\ul{v}$ be a force field; how much work is done by $\ul{v}$ moving a point mass
along L from A to B?

\textbf{A}: Split $L$ into infinitesimal line segments $d\ul{r}$.
Because they are infinitesimal, the $d\ul{r}$ are straight and $\ul{v}$ is homogeneous within a single $d\ul{r}$.
The work done by the force field $\ul{v}$ per line segment is
\[
	dW = \ul{v} \cdot d\ul{r}
\]

and the total work correspondingly is
\[
	W = \int_L \ul{v} \cdot d\ul{r}
\]

For parametric curves
\[
	L: t \mapsto c(t) \quad \Rightarrow \quad d\ul{r} \simeq \dot{c}(t) dt
\]




\subsection*{Gauss Theorem}
Consider a vector field $\ul{v}$ that is defined and continuously differentiable in a closed region $B$ with boundary $\partial B$ and outer unit normal $\ul{n}$.

\[
	\color{Plum}
	\oint_{\partial B} \ul{v} \cdot \ul{n} \, dS  =  \int_{B} \div \ul{v}  \, dV =  \int_{B} \nabla \cdot \ul{v}  \, dV
\]



\subsection*{Stokes Theorem}
Consider a vector field $\ul{v}$ that is defined and continuously differentiable in a region $D$.
Consider further a bounded surface S that is entirely contained in D and has the border line $\partial S$
$C$ is a path along $\partial S$ such that its sense forms a right-hand screw with the normal onto $S$.

\[
	\color{Plum}
	\oint_{C} \ul{v} \cdot d\ul{r}  =  \int_{S} \curl \ul{v}  \cdot \ul{n} \, dS =  \int_{S} (\nabla \times \ul{v})  \cdot \ul{n} \, dS
\]



\subsection*{Greens Theorems}
Obtained by applying Gauss to $\ul{v} = f_1 \nabla f_2$

\[
\color{Plum}
\begin{aligned}
	\int_{B} (f_1 \Delta f_2  + \nabla f_1 \nabla f_2) \, dV = & \oint_{\partial B} f_1 \nabla f_2 \cdot \ul{n} \, dS \\
	\\
	\int_{B} (f_1 \Delta f_2  - f_2 \Delta f_1 ) \, dV = &\oint_{\partial B} (f_1 \nabla f_2 -  f_2 \nabla f_1) \cdot \ul{n} \,dS \\
\end{aligned}
\]



\subsection*{Conservative Fields}
A vector field $\ul{v}$ is called conservative if and only if the work along all possible \ul{paths} from $P$ to $Q$ is equal, for all $P, Q \in \mathbb{R}^n$.

\begin{comment}
The work does not depend on the specific path chosen, but only on the starting point and end point.
These points ($P$ and $Q$ above) are hence states of the system, which we earlier identified with the levels of the reservoirs.
For a conserved quantity, it does hence not matter how the level of a reservoir is reached, but only how large it is.
\end{comment}

\begin{enumerate}
	[label=$\bullet$]
	\item Each gradient field is conservative and vice versa, and thus also called potential field. 
	\item Each gradient field is vortex free and vice versa, i.e. $\nabla \times \nabla f \equiv 0$ .
\end{enumerate}


The definition is equivalent to
\[
\begin{aligned}
	\ul{v} \text{ conservative} &\Leftrightarrow \curl \ul{v} = \nabla \times v = 0 \\
	\ul{v} \text{ conservative} &\Leftrightarrow \exists f : \mathbb{R}^n \to \mathbb{R} : \nabla f = \ul{v} \\
\end{aligned}
\]









\begin{comment}
\subsection*{Exercise}

\[
\begin{aligned}
& \text { curl curl }=\text { curl }\left(\begin{array}{l}
\frac{\partial}{\partial y} v_3-\frac{\partial}{\partial z} v_2 \\
\frac{\partial}{\partial z} v_1-\frac{\partial}{\partial x} v_3 \\
\frac{\partial}{\partial x} v_2-\frac{\partial}{\partial y} v_1
\end{array}\right) \\
& =\left(\begin{array}{l}
\frac{\partial}{\partial x} \\
\frac{\partial}{\partial y} \\
\frac{\partial}{\partial z}
\end{array}\right) \times\left(\begin{array}{l}
\frac{\partial}{\partial y} v_3-\frac{\partial}{\partial z} v_2 \\
\frac{\partial}{\partial z} v_1-\frac{\partial}{\partial x} v_3 \\
\frac{\partial}{\partial x} v_2-\frac{\partial}{\partial y} v_1
\end{array}\right) \\
& \left.=\left(\begin{array}{l}
\left(\frac{\partial^2}{\partial x \partial y} v_2-\frac{\partial^2}{\partial y^2} v_1\right)-\left(\frac{\partial^2}{\partial z^2} v_1-\frac{\partial^2}{\partial x \partial z} v_3\right. \\
\left.\frac{\partial^2}{\partial z \partial y} v_3-\frac{\partial^2}{\partial z^2} v_2\right)-\left(\frac{\partial^2}{\partial x^2} v_2-\frac{\partial^2}{\partial x \partial y} v_1\right. \\
\left.\frac{\partial^2}{\partial x \partial z} v_1-\frac{\partial^2}{\partial x^2} v_3\right)-\left(\frac{\partial^2}{\partial y^2} v_3-\frac{\partial^2}{\partial y \partial z} v_2\right.
\end{array}\right)\right) \\
& =\left(\begin{array}{l}
\frac{\partial^2}{\partial x \partial y_2} v_2+\frac{\partial^2}{\partial x \partial z} v_3-\frac{\partial^2}{\partial y^2} v_1-\frac{\partial^2}{\partial z^2} v_1+\frac{\partial^2}{\partial x \partial x} v_1-\frac{\partial^2}{\partial x^2} v_1 \\
\frac{\partial^2}{\partial z \partial y^2} v_3+\frac{\partial^2}{\partial x \partial y} v_1-\frac{\partial^2}{\partial y^2} v_2-\frac{\partial^2}{\partial \partial^2} v_2+\frac{\partial^2}{\partial y \partial y} v_2-\frac{\partial^2}{\partial y^2} v_2 \\
\frac{\partial^2}{\partial x \partial z} v_1+\frac{\partial^2}{\partial y \partial z} v_2-\frac{\partial^2}{\partial y^2} v_3-\frac{\partial^2}{\partial z^2} v_3+\frac{\partial^2 y}{\partial z \partial z} v_3-\frac{\partial^2}{\partial z^2} v_3
\end{array}\right) \\
& =\left(\begin{array}{l}
\frac{\partial^2}{\partial x \partial y} v_2+\frac{\partial^2}{\partial x \partial z} v_3+\frac{\partial^2}{\partial x^2} v_1-\Delta v_1 \\
\frac{\partial^2}{\partial z \partial \partial v_3}+\frac{\partial^2}{\partial x \partial y} v_1+\frac{\partial^2}{\partial y^2} v_2-\Delta v_2 \\
\frac{\partial^2}{\partial x \partial z} v_1+\frac{\partial^2}{\partial y \partial z} v_2+\frac{\partial^2}{\partial z^2} v_3-\Delta v_3
\end{array}\right) \\
& =\left(\begin{array}{c}
\frac{\partial}{\partial x}\left(\frac{\partial}{\partial x} v_1+\frac{\partial}{\partial y} v_2+\frac{\partial}{\partial z} v_3\right) \\
\frac{\partial}{\partial y}\left(\frac{\partial}{\partial x} v_1+\frac{\partial}{\partial y} v_2+\frac{\partial}{\partial z} v_3\right) \\
\frac{\partial}{\partial z}\left(\frac{\partial}{\partial x} v_1+\frac{\partial}{\partial y} v_2+\frac{\partial}{\partial z} v_3\right)
\end{array}\right)-\Delta \mathbf{v} \\
& =\left(\begin{array}{l}
\frac{\partial}{\partial x} \\
\frac{\partial}{\partial y} \\
\frac{\partial}{\partial z}
\end{array}\right)\left(\frac{\partial}{\partial x} v_1+\frac{\partial}{\partial y} v_2+\frac{\partial}{\partial z} v_3\right)-\Delta \mathbf{v} \\
& =\operatorname{grad} d i v \mathbf{v}-\Delta \mathbf{v} \\
&
\end{aligned}
\]

\begin{comment}













\begin{comment}


\begin{aligned}
	\grad(f_1 f_2) = & f_1 \grad( f_2) + f_2 \grad(f_1 )\\
	\gradF(f) = & F'(f) \grad(f) \\
	\div f\ul{v} = &  \ul{v} \grad( f) + f \div \ul{v}\\
	\div \ul{v} = & \nabla \cdot \ul{v}\\
	\curl \ul{v}_1 + \ul{v}_2 = &  \curl \ul{v}_1 + \curl \ul{v}_2\\
	\curl c\ul{v} = &  c \curl \ul{v} \\
	\curl f \ul{v} = &  f \curl\ul{v}  -  \ul{v} \times \grad(f)\\
	\div\curl \ul{v} = &  0 \\
	\curl\grad f = &  \ul{0} \\
	\div\grad f = & \Delta f \\
	\curl\curl \ul{v} = & \grad  \div \ul{v} - \Delta \ul{v} \\
	\div (\ul{v}_1 \times \ul{v}_2) = & \ul{v}_2 \cdot \curl\ul{v}_1 - \ul{v}_1 \cdot \curl\ul{v}_2 \\
\end{aligned}


1. Wenn $F \wedge G$ eine Tautologie ist, dann ist $F$ eine Tautologie und $G$ auch.
2. Umgekehrt: Sind $F$ und $G$ Tautologien, dann ist auch $F \wedge G$ eine.
\emph{Beweis.}
1. Annahme: $F \wedge G$ sei eine Tautologie.
Dann: Für jede Belegung $B$ wertet $F \wedge G$ zu wahr aus.
Dann: Das ist nur der Fall, wenn sowohl $F$ als auch $G$ (für jedes $B$) zu wahr auswerten.
Dann: Für jede Belegung $B$ wertet $F$ zu wahr aus. Und:
Für jede Belegung $B$ wertet $G$ zu wahr aus.
Dann: $F$ ist Tautologie und $G$ ist Tautologie.
2. Annahme: $F$ ist Tautologie und $G$ ist Tautologie.
Dann: Für jede Belegung $B_1$ wertet $F$ zu wahr aus. Und: Für jede Belegung $B_2$ wertet $G$ zu wahr aus.
Dann: Für jede Belegung $B$ wertet $F \wedge G$ zu wahr aus.
Dann: $F \wedge G$ ist eine Tautologie.
\subsection*{Äquivalenz und Folgerung}
$p\equiv q$ gilt genau dann, wenn sowohl $p\models q$ als auch $q\models p$ gelten. \emph{Beweis.}
$p\equiv q$ GDW $p\Leftrightarrow q$ ist Tautologie nach Def. von $\equiv$
GDW $(p\Rightarrow q) \wedge (q\Rightarrow p)$ ist Tautologie
GDW $(p\Rightarrow q)$ ist Tautologie und $(q\Rightarrow p)$ ist Tautologie
GDW $(p\models q)$ gilt und $q\models p$ gilt.
\subsection*{Substitution}
Ersetzt man in einer Formel eine beliebige Teilformel $F$ durch eine logisch äquivalente
Teilformel $F'$, so verändert sich der Wahrheitswerteverlauf der Gesamtformel nicht.
Man kann Formeln also vereinfachen, indem man Teilformeln durch äquivalente
(einfachere) Teilformeln ersetzt.
\subsection*{Universum}
Die freien Variablen in einer Aussagenform können durch Objekte aus einer als
Universum bezeichneten Gesamtheit wie $\mathbb{N},\mathbb{R},\mathbb{Z},\mathbb{Q}$ ersetzt werden.
\subsection*{Tautologien}
$(p\wedge q)\Rightarrow p$\text{ bzw. }$p\Rightarrow (p\vee q)$\\
$(q\Rightarrow p)\vee (\neg q\Rightarrow p)$\\
$(p\Rightarrow q)\Leftrightarrow (\neg p\vee q)$\\
$(p\Rightarrow q)\Leftrightarrow (\neg q\Rightarrow\neg p)$ \hfill\text{(Kontraposition)}\\
$(p\wedge (p\Rightarrow q))\Rightarrow q$ \hfill\text{(Modus Ponens)}\\
$((p\Rightarrow q)\wedge (q\Rightarrow r))\Rightarrow (p\Rightarrow r)$\\
$((p\Rightarrow q)\wedge (p\Rightarrow r))\Rightarrow (p\Rightarrow (q\wedge r))$\\
$((p\Rightarrow q)\wedge (q\Rightarrow p))\Leftrightarrow (p\Leftrightarrow q)$
\subsection*{Nützliche Äquivalenzen}
Kommutativität:\\
$(p \wedge q) \equiv (q \wedge p)$\\
$(p \vee q) \equiv (q \vee p)$\\
Assoziativität:\\
$(p \wedge (q \wedge r)) \equiv ((p \wedge q) \wedge r)$\\
$(p \vee (q \vee r)) \equiv ((p \vee q) \vee r)$\\
Distributivität:\\
$(p \wedge (q \vee r)) \equiv ((p \wedge q) \vee (p \wedge r))$\\
$(p \vee (q \wedge r)) \equiv ((p \vee q) \wedge (p \vee r))$\\
Idempotenz:\\
$(p \wedge p) \equiv p$\\
$(p \vee p) \equiv p$\\
Doppelnegation:\\
$\neg (\neg p) \equiv p$\\
de Morgans Regeln:\\
$\neg (p \wedge q) \equiv ((\neg p) \vee (\neg q))$\\
$\neg (p \vee q) \equiv ((\neg p) \wedge (\neg q))$\\
Definition Implikation:\\
$(p \Rightarrow q) \equiv (\neg p \vee q)$\\
Tautologieregeln:\\
$(p \wedge q) \equiv p$\hfill (falls $q$ eine Tautologie ist)\\
$(p \vee q) \equiv q$\\
Kontradiktionsregeln:\\
$(p \wedge q) \equiv q$\hfill (falls $q$ eine Kontradiktion ist)\\
$(p \vee q) \equiv p$\\
Absorptionsregeln:\\
$(p \wedge (p \vee q)) \equiv p$\\
$(p \vee (p \wedge q)) \equiv p$\\
Prinzip vom ausgeschlossenen Dritten:\\
$p \vee (\neg p) \equiv w$\\\
Prinzip vom ausgeschlossenen Widerspruch:\\
$p \wedge (\neg p) \equiv f$
\subsection*{Äquivalenzen von quant. Aussagen}
Negationsregeln:\\
$\neg\forall x:p(x)\equiv\exists x:(\neg p(x))$\\
$\neg\exists x:p(x)\equiv\forall x:(\neg p(x))$\\
Ausklammerregeln:\\
$(\forall x:p(x)\wedge\forall y:q(y))\equiv\forall z:(p(z)\wedge q(z))$\\
$(\exists x:p(x)\wedge\exists y:q(y))\equiv\exists z:(p(z)\wedge q(z))$\\
Vertauschungsregeln\\
$\forall x\forall y:p(x,y)\equiv\forall y\forall x:p(x,y)$\\
$\exists x\exists y:p(x,y)\equiv\forall y\exists x:p(x,y)$
\subsection*{Äquivalenzumformung}
Wir demonstrieren an der Formel $\neg (\neg p \wedge q) \wedge (p \vee q)$, wie man mit Hilfe der
aufgelisteten logischen Äquivalenzen tatsächlich zu Vereinfachungen kommen kann:\\
$\phantom{{}\equiv{}} \neg (\neg p \wedge q) \wedge (p \vee q)$\\
$\equiv (\neg (\neg p) \vee (\neg q)) \wedge (p \vee q)$\hfill de Morgan\\
$\equiv (p \vee (\neg q)) \wedge (p \vee q)$\hfill Doppelnegation\\
$\equiv p \vee ((\neg q) \wedge q)$\hfill Distributivtät v.r.n.l.\\
$\equiv p \vee (q \wedge (\neg q))$\hfill Kommutativtät\\
$\equiv p \vee f$\hfill Prinzip v. ausgeschl. Widerspruch\\
$\equiv p$\hfill Kontradiktionsregel

\end{comment}




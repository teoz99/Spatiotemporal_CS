\definecolor{myblue}{cmyk}{1,.72,0,.38}



\def\firstcircle{(0,0) circle (1.5cm)}
\def\secondcircle{(0:2cm) circle (1.5cm)}

\colorlet{circle edge}{myblue}
\colorlet{circle area}{myblue!5}

\tikzset{filled/.style={fill=circle area, draw=circle edge, thick},
    outline/.style={draw=circle edge, thick}}


%Own Operators----------------------------------------------------------------------------------------------------------------------------------------------------------------------------------------------------
\DeclareMathOperator*{\argmax}{arg\,max}
\DeclareMathOperator*{\argmin}{arg\,min}
\DeclareMathOperator*{\curl}{curl}

\let\div\relax %remove the existing div command to allow operator definition
\DeclareMathOperator*{\div}{div}

\DeclareMathOperator*{\grad}{grad}



%Own Commands----------------------------------------------------------------------------------------------------------------------------------------------------------------------------------------------------
%1. Inequality in set: ieset
%Default: lhs is tau
\newcommand{\ieset}[2][\tau]{\{#1 \le #2\}} 
%Example: to get {X <= t} use this expression $\ieset[X]{t}$

%2. Conditional Expectation for Sigma Algebra: scEx
%Default: sigma Algebra is F_n
\newcommand{\scEx}[2][n]{E[#2 \mid \mathfrak{F}_#1]}
%Example: to get E[ X | F_tau ] use this expression $\scEx[\tau]{X}$

%3. In curly brackets: icb
\newcommand{\icb}[1]{\{ #1 \}}
%Example: to get {a = b} use this expression $\icb{a = b}

%4. Norm: norm
\newcommand{\norm}[1]{\left\lVert#1\right\rVert}
%Example: to get ||X|| use the expression \norm{X}

%5. Column Vector
\newcommand*\colvec[1]{\begin{pmatrix}#1\end{pmatrix}}
%Example: \colvec{a \\ b} gives vector (a,b) transposed.

%6.Row Vector
\newcommand{\rvec}[1]{\begin{bmatrix} #1 \end{bmatrix}}
%Example: \rvect{a & b} gives vector (a,b)




\pgfdeclarelayer{background}
\pgfsetlayers{background,main}


\everymath\expandafter{\the\everymath \color{myblue}}


\renewcommand{\baselinestretch}{.8}
\pagestyle{empty}


\global\mdfdefinestyle{header}{%
linecolor=gray,linewidth=1pt,%
leftmargin=0mm,rightmargin=0mm,skipbelow=0mm,skipabove=0mm,
}


\newcommand{\header}{
\begin{mdframed}[style=header]
\footnotesize
\sffamily
Cheat Sheet\\
Page \thepage
\end{mdframed}
}


\makeatletter % Author: https://tex.stackexchange.com/questions/218587/how-to-set-one-header-for-each-page-using-multicols
\renewcommand{\section}{\@startsection{section}{1}{0mm}%
                                {.2ex}%
                                {.2ex}%x
                                {\color{myblue}\sffamily\small\bfseries}}
\renewcommand{\subsection}{\@startsection{subsection}{1}{0mm}%
                                {.2ex}%
                                {.2ex}%x
                                {\sffamily\bfseries}}




\makeatother
\setlength{\parindent}{0pt}





\section{Modeling Dynamics}

\subsection*{Modeling Definitions}
\textbf{System}: potential source of data, defined by boundary, inputs and outputs.
\textbf{Experiment}: process of extracting datafrom a system \textrightarrow{} observes trajectory of the sys inputs;applies change to the sys inputs and records the reaction in the outputs.
\textbf{Model}:anything to which the experiment can be applied in order to answer questions about the system \textrightarrow{} model is a sys;not always mathematical;defined on (sys,exp).
\textbf{Simulation}: exp on a model
\textbf{Modeling}:process of organizing knowledge about sys.

\subsection*{Continuity assumption}

\textbf{Particles}:modeled by $x_p(t)$ position, $\omega(P,t)$ property.
\textuptarrow{} P \textrightarrow{} continuous field.
Averaging volume $V$ \textrightarrow{} concentration $u$. Increasing $V$:
\begin{comment}

\end{comment}
\textbf{\lambda (continuum limit)}: dynamic lim; below it, $V$ too small \textrightarrow average strongly fluctuates
\textbf{$L$ length scale of field variations}: spatiotemporal limit, related to Kuramoto length; above it, space must be taken into accout. If \lambda \textless property \textless $L$ \textrightarrow non-spatial dynamics domain. 
$\textbf{K_n}=\frac{\lambda}{L}$ Knudsen number
\textbf{Extensive} property: \lambda \textless value \textless $L$ depends on V, else \textbf{intensive}.

\subsection*{Dimensional analysis}
\textbf{Dimensional analysis}: technique to analyze sys by estimating orders of magnitude, finding correlations between variables, checking similitude between model and sys and consistency of equations.

All dimensions are power series of 6 \textbf{indipendent dimensions}: mass M, time T, temperature (\theta), charge (C) and luminosity (J).

\textbf{Dimensionless grouping}: algebraic combo of variables that has no dimension.
Given $n$ indpendent dimensions, $p$ variables \textrightarrow $p-n$ dimensionless groupings.

\textbf{Taylor's Method}:
\begin{enumerate}[label=$\bullet$]
	\item determine $p-n$ groupings
	\item write matrix(variables, indip 					dimensions)
	\item choose column and row with less non-		zero entries and eliminate them with 			Gaussian elimination (the variables which 		rows are used for Gauss elim must be 			divided by the variable of the eliminated 		row)
	\item if only one var is left for a 				certain dim, delete it (and the 				corresponding dim)
	\item repeat gauss elim until all etries 		are 0
\end{enumerate}

\textbf{Dynamic similitude}: when all dimensionless groupings of model and reality match in value. The model is built s.t. the test results are applicable to reality.

\subsection{Time scales}
Dynamics \textrightarrow time scales:  \textbf{relevant}, as \textbf{differential equations}; \textbf{slow}, as \textbf{constants}; \textbf{fast}, as \textbf{algebraic terms}

\textbf{Stability}: an integrator is stable if all trajectories of a stable sys remain bounded for $t$ \textrightarrow $\infty$.
e.g. \textbf{Euler scheme}: $x_{t+1}=x_{t}-hax_{t}=[1-ha]^{t+1}x_0$ stable if \textrightarrow $|x_{t+1}| \textless C$ \textrightarrow $|1-ha| \textless 1$.
In general: $x_{t+1}=Fx_{t}$ stable if eigenvalues \lambda of F with Re(\lambda) \textless 0 have $|\lambda|$ \textless 1.
\textbf{Stiffness}: if $\frac{\lambda_{max}}{\lambda_{min}}$ is large.

\subsection{Reservoirs and Flows}
\textbf{Reservoir} is the storage of a sys, its levels are called \textbf{state variables}. The levels can be only extensive quantities. If all levels are known, the behavior of the sys can be predicted. The levels map the state space.
\textbf{Flows} connect reservoirs.

\textbf{Modeling steps}:
\begin{enumerate}[label=$\bullet$]
	\item define sys boundaries, inputs,       	outputs
	\item identify reservoirs (relevant t) and 
	their levels
	\item formulate algebraiìc equations that 		describe flows
	\item formulate the balance equations
	that govern the dynamics of the reservoir 		levels
	\item simplify equations, non dimensionalize...
	\item solve equations to obtain level(t)
	\item parameter identification
	\item validate model on external data
\end{enumerate}

\textbf{Casuality diagrams}: a way of visualizing casual relationship in a sys/model. 
Conventions: 
\begin{enumerate}[label=$\bullet$]
	\item reservoir: shaded box
	\item flow:plain box
	\item casuality: arrow indicating variable
	\item sum: node
\end{enumerate}

\begin{comment}

\subsection{Dynamic similitude example}
\textbf{Dimensional analysis example}

variables: F=[$MLT_{-2}$], \mu=[$ML_{-1}T_{-1}$], \rho=[$ML_{-3}$], U=[$LT_{-1}$], D=[L]. n=3, p=5. 

$ \begin{bmatrix}
 & M & L & T \\
\mu & 1 & -1 & -1 \\
\rho & 1 & -3 & 0 \\
U & 0 & 1 & -1 \\
D & 0 & 1 & 0 \\
F & 1 & 1 & -2 
\end{bmatrix}  $ \underrightarrow{\text{R1-R2,R5-R2}}$\begin{bmatrix}

& M & L & T \\
\frac{\mu}{\rho} & 0 & 2 & -1 \\
\rho & 1 & -3 & 0 \\
U & 0 & 1 &-1 \\
D & 0 & 1 & 0 \\
\frac{F}{\rho} & 0 & 4 & -2 
\end{bmatrix}  $ \underrightarrow{\text{delete M col and \rho row}}$\begin{bmatrix}

& L & T \\
\frac{\mu}{\rho} & 2 & -1 \\
U & 1 & -1 \\
D & 1 & 0 \\
\frac{F}{\rho} & 4 & -2 
\end{bmatrix}  $ \underrightarrow{\text{R2-R1,R4-2R1}}

\\

$ \begin{bmatrix}
& L & T \\
\frac{\mu}{\ro} & 2 & -1 \\
\frac{\rho U}{\mu} & -1 & 0 \\
D & 1 & 0 \\
\frac{\rho F}{\mu^{2}} & 0 & 0 
\end{bmatrix}  $ \underrightarrow{\text{eliminate T and \frac{\mu}{\rho}}} $ \begin{bmatrix}
& L \\
\frac{\rho U}{\mu} & -1\\
D & 1\\
\frac{\rho F}{\mu^{2}} & 0 
\end{bmatrix}  $

\end{comment}
Navier Stokes: $\rho\pdv[]{u}{t} + \rho u\pdv[]{u}{x} = -\pdv[]{\rho}{x} + \mu\pdv[2]{\mu}{x}$. Sound source L moving at constant velocity U, emitting tone with freq $ \omega$. Fluid density far away from source ($\rho_{\infty}$)

\vspace{1mm}

Dimensionless variables:
$\overline{x}=\frac{x}{L},\overline{t}=\omega t, \overline{u}=\frac{u}{U}, \overline{\rho}=\frac{\rho}{\rho_{\infty}},\overline{p}=\frac{p}{\rho_{\infty} U^2}$

\vspace{1mm}

Express quantities by the dimensionless quantities:
  $ \overline{\rho}\rho_{\infty}\pdv[]{\overline{u} U}{\frac{\overline{t}}\omega{}} + \overline{\rho}\rho_{\infty}\overline{u} U\pdv[]{\overline{u} U}{\overline{x} L} = -\pdv[]{\overline{p}\rho_{\infty} U^2}{\overline{x} L} + \mu\pdv[2]{\overline{\mu} U}{\overline{x}^2 L}   \vspace{1mm} \longrightarrow   \rho_{\infty} U\omega\overline{\rho}\pdv[]{\overline{u}}{\overline{t}} +  \frac{\rho_{\infty} U^2}{L} \overline{\rho}\overline{u}\pdv[]{\overline{u}}{\overline{x}} = -\frac{\rho_{\infty} U^2}{L}\pdv[]{\overline{p}}{\overline{x} } + \frac{\mu U}{L^2}\pdv[2]{\overline{\mu}}{\overline{x}}$
\vspace{1mm}

Assuming a sound source $\overline{U}=10U$, to ensure dynamic similitude: $\overline{\omega}=10\omega$ and $\overline{\rho}=0.1\rho_{\infty}$ or $\overline{\mu}=10\mu$

\subsection{Continuum assumption pseudocode}
Take averaging volumes of increasing size, centered at the box center and count the number of particles inside:

\begin{algorithm}[H]
			
    \caption*{Continuum assumption}
    \begin{algorithmic}
			\State Box length L
			\State Create matrix Particles(nx3) of random values in 				the range (-L,L)
			\State plot Particles
			\State Set number of averaging boxes NumAvBoxes, max length of the boxes maxavBoxLen, vector avBoxLen, volume avBoxVol
            \For{i in NumAvBoxes}
                
                    \State  Current particles matrix \										Currparticles=find Particles(i)\textless 								avBoxLen(i)
                                           
            \EndFor
            \State Plot (avBoxVol,size(Currparticles)/avBoxVol)
				
    \end{algorithmic}
			
\end{algorithm}





